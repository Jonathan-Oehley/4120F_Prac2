% !TeX spellcheck = en_GB 
\section{Methodology}
The implementation of a median filter requires some definition around the pixels near the edge of an image. The outer edge-case pixels were calculated by the following means; the area over which the pixel component values were compared with to determine the median was simply truncated by the edge of the image. In other words, for the pixels in the furthest corners, the area over which the median was determined was 5 x 5 pixels.

\subsection{Golden Measure}
The golden measure was created as a relatively simple block of code that's main objective was to be a working model. From this, further implementation could be compared and a comparison drawn.

The golden measure sorting method was implemented as a simple bubble sort. While not the quickest sorting technique, the bubble sort is easy to code and understand. Therefore, it was possible to implement the sorting method relatively quickly in this manner. The filter was simply implemented by flattening the 2 dimensional comparison area, into a single array before passing it to the sorting algorithm to determine the median.

\subsection{Multi-thread Implementation}
This implementation looked to improve the execution time of the golden measure. This was done using two techniques. 

Firstly, instead of implementing a bubble sort by default, a investigation was made between bubble sort, a select sort function (which only sorted just above halfway to get the median) and the hybrid sort function (std::sort()) built into c++. The fastest sorting method was used in this implementation which will be discussed in the Results section.

Secondly, the golden measure implementation was converted into a multi-threaded application. This was done through data-partitioning of the image into rows. This allows the data sorting tasks to be split up over multiple processors to decrease the overall execution time of the application
  
\subsection{Experiment Procedure}
The program was coded using a shared git repository to allow collaboration between the two students. The code was tested periodically to ensure functionality of each function.

Testing was conducted by running the program a few times to ensure that it was stored in cache before taking readings. Then the program was run with different inputs, number of threads and sorting functions. The execution time for each case was measured for multiple runs of the program and averaged across them. This generates a single value for the execution time of the program with those parameters.

A number of jpg images were used for testing. These were of various sizes and are described in Table~\ref{tab:Image dimensions}. These are of note as the resulting execution times were directly related to the size of the image.

\Table{Image dimensions\label{tab:Image dimensions}}{lccr}{ 
	\textbf{} & \textbf{Height} & \textbf{Width} & \textbf{Pixels E+3} 
}{
	Greatwall.jpg & 2560 & 1920 & 4915 \\
	Fly.jpg & 1024 & 821 & 841 \\
	Alan.jpg & 640 & 640 & 410 \\	
	Small.jpg & 304 & 300 & 91 \\
}
{Example}

